\section{Les premiers pas sous Beamer}

  		\subsection{Les paquets pour faire des diapositives}

	Il existe deux paquets pour faire des paquets sous \LaTeX~:
\begin{description}
	\item[prosper] Nous n'en parlerons pas ici
	\item[beamer] c'est le paquet dont je vous parlerais ici
\end{description}
		
	Beamer a �t� utilis� pour faire ces diapositives. Vous pouvez donc regarder les sources des documents pour avoir un exemple plus vivant.


  		\subsection{Dans le pr�ambule...}

	Premi�rement, il faut changer le type du document, au lieu de \emph{article} par exemple il faut mettre~:
\code
\documentclass{beamer}
	\usetheme{Warsaw}
\end{Verbatim}

	La premi�re ligne correspond au type \emph{beamer} qui nous int�resse ici. Le second permet de d�finir le th�me, c'est-�-dire l'apparence de vos diapos. Ce sont des "mod�les" de diapos. Vous pourrez �videmment personnaliser si vous le souhaitez.
		


  		\subsection{Dans le corps du texte...}

Dans le corps du texte, les commandes � utiliser pour d�finir une diapositive sont les suivantes~:
\code
\begin{frame}
Mon texte...
\end{frame}
\end{Verbatim}

	ce qui donne la diapo de la page suivante...


Mon texte...


\section{Titres et organisation}


	Les commandes \emph{\textbackslash section} et \emph{\textbackslash subsection} sont disponibles et vous permettent de structurer votre pr�sentation. 

\vspace{0.1cm}
Certains mod�les utilisent ces informations pour cr�er un affichage de la progression de la pr�sentation.



  		\subsection{Titres des diapositives}

Vous pouvez rajouter des titres aux diapositives. Ces titres seront clairement visibles sur le document mais ne seront pas index�s dans la table des mati�res.

\vspace{0.1cm}
Sur ce mod�le il occupe le bandeau juste au dessus du texte. La syntaxe est relativement simple~:

\code
\begin{frame}
\frametitle{Mon titre}
Mon texte ....
\end{frame}
\end{Verbatim}


Mon texte ....


  		\subsection{Titres des diapositives}

La syntaxe ci-dessous est aussi possible... pour le m�me r�sultat.

\code
\begin{frame}{Mon titre}
\frametitle
Mon texte ....
\end{frame}
\end{Verbatim}



  		\subsection{Ajout de Verbatim}

S'il y a au moins un bloc \emph{listings} ou \emph{verbatim} il faut ajouter une balise pour \LaTeX~:

\code
\begin{frame}[containsverbatim]
\frametitle{Mon titre}
Mon texte ....
\end{frame}
\end{Verbatim}



\section{Mise en page et environnements}

  		\subsection{Ajout de Verbatim}

On ne peut pas utiliser la syntaxe suivante dans cet ordre.

\code
\begin{frame}[containsverbatim]{Mon titre}
Mon texte ....
\end{frame}
\end{Verbatim}



  		\subsection{Options de mise en page suppl�mentaires...}

Beamer offre la possibilit� de mettre tout ou partie du texte en �vidence en utilisant la balise \emph{block}

\code
\begin{frame}[containsverbatim]
\frametitle{Mon titre}
\begin{block}{Mon titre de bloc}
Mon texte ....
\end{block}
\end{frame}
\end{Verbatim}


  		\subsection{Options de mise en page suppl�mentaires...}

Les environnements  suivants sont �galement disponibles...

\begin{itemize}
	\item alertblock
	\item theorem
	\item proof
	\item example
	\item beamercolorbox
\end{itemize}




Comme d'autres logiciels, on peut ins�rer des �tapes~:

\code
\begin{itemize}
	\item<1-> alertblock
	\item<2-> theorem
	\item<3-> proof
	\item<4-> example
	\item<5-> beamercolorbox
\end{itemize}
\end{Verbatim}




Dans le code pr�c�dent, \emph{alertblock} sera visible de la diapo un � la derni�re, \emph{theorem}, de la diapo 2 � la derni�re et ainsi de suite...


\section{Th�mes et sous forme d'article...}

  		\subsection{Les th�mes}

Pour avoir un aper�u des th�mes Beamer, il y a ce \href{http://mcclinews.free.fr/latex/beamergalerie.php}{site }.




  		\subsection{Sortie sous forme d'article}

Pour sortir les diapositives sous la forme d'un article, il suffit de changer le \emph{documentclass}~:

\code
\documentclass{article}
...
\usepackage{beamerarticle}
...
\end{Verbatim}




  					


