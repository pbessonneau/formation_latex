\documentclass{beamer}
\usetheme[compress]{Singapore}
\useoutertheme{miniframes}

% Pour les documents en fran�ais...
	\usepackage[latin1]{inputenc}
	\usepackage[french]{babel}    
	\usepackage[french]{varioref} 
	
% Math�matiques
	\usepackage{amsmath}
	
% A documenter	
	\usepackage{moreverb}
	\usepackage{lipsum}

% Pour ins�rer des graphiques	
	\usepackage{eso-pic,graphicx}	% Graphique simples    
	\usepackage{subfigure}			% Graphiques multiples
	\usepackage{xcolor}
	\usepackage{tikz}
	\usetikzlibrary{positioning}						
	
% Pour ins�rer des couleurs	
	\usepackage{color}

% Outil suppl�mentaire pour les tableaux
	\usepackage{multirow}
 	\usepackage{booktabs}
	\usepackage{longtable}
	\usepackage{colortbl}
	
% Rotation des objets et des pages
	\usepackage{rotating}
	\usepackage{lscape}

% Pour ins�rer du code source, LaTeX ou SAS par exemple.
	\usepackage{verbatim}
	\usepackage{fancyvrb}
	\usepackage{listings} 

% Pour ins�rer des hyperliens
	\usepackage{hyperref}

% American Psychological Association (for bibliographic references).
	\usepackage{apacite}
  
% Pour l'utilisation des macros
	\usepackage{xspace}

% Pour l'utilisation de notes en fin de document.
	\usepackage{endnotes}

% Rotation
	\usepackage{rotating}

% Pour les t�ches de caf�
	\usepackage{coffee}

% Symboles suppl�mentaires
	\usepackage{bbding}
	\usepackage{pifont}

% Pour les listes num�rot�es
	\usepackage{enumerate}

% Pour la derni�re page
	\usepackage{lastpage}

% Pour Highlight d'Andre Simon
\usepackage{alltt}

% pour les symboles
\usepackage{keystroke}
%\usepackage{feyn}
\usepackage{bbding}
\usepackage{phonetic}

% Pour ins�rer des dessins de Linux
\newcommand{\LinuxA}{\includegraphics[height=0.5cm]{Graphiques/linux.png}}
\newcommand{\LinuxB}{\includegraphics[height=0.5cm]{Graphiques/linux.png}\xspace}

% Macro pour les petits dessins pour les diff�rents OS.
\newcommand{\Windows}{\emph{Windows}\xspace}
\newcommand{\Mac}{\emph{Mac OS X}\xspace}
\newcommand{\Linux}{\emph{Linux}\xspace}
\newcommand{\MikTeX}{MiK\tex\xspace}

% Des raccourcis pour les commandes \LaTeX, \TeX, ...
\newcommand{\latex}{\LaTeX\xspace}
\newcommand{\latexe}{\LaTeXe\xspace}
\newcommand{\tex}{\TeX\xspace}

% Commande pour le mode Verbatim
\newcommand{\code}{\vspace{0.2cm}\begin{Verbatim}[frame=single,label=Code,fontsize=\small]}
\newcommand{\tinycode}{\vspace{0.2cm}\begin{Verbatim}[frame=single,label=Code,fontsize=\tiny]}

% From Framabook (www.framasoft.net)
\newcommand{\latexcom}[1]{{\mdseries\ttfamily\upshape\symbol{92}#1}}
\newcommand{\indexcom}[1]{%
  \index{#1@\protect\texttt{\symbol{92}#1}}}
\newcommand{\ltxcom}[1]{%
  \latexcom{#1}\indexcom{#1}}  

\newcommand{\hlstd}[1]{\textcolor[rgb]{0,0,0}{#1}}
\newcommand{\hlnum}[1]{\textcolor[rgb]{0.5,0,0.5}{\bf{#1}}}
\newcommand{\hlesc}[1]{\textcolor[rgb]{1,0,1}{\bf{#1}}}
\newcommand{\hlstr}[1]{\textcolor[rgb]{0.65,0.52,0}{#1}}
\newcommand{\hlpps}[1]{\textcolor[rgb]{0,0,1}{#1}}
\newcommand{\hlslc}[1]{\textcolor[rgb]{0.95,0.47,0}{#1}}
\newcommand{\hlcom}[1]{\textcolor[rgb]{1,0.5,0}{#1}}
\newcommand{\hlppc}[1]{\textcolor[rgb]{0,0.5,0.75}{\bf{#1}}}
\newcommand{\hlopt}[1]{\textcolor[rgb]{1,0,0.5}{\bf{#1}}}
\newcommand{\hlipl}[1]{\textcolor[rgb]{0.62,0.36,1}{#1}}
\newcommand{\hllin}[1]{\textcolor[rgb]{0.19,0.19,0.19}{#1}}
\newcommand{\hlkwa}[1]{\textcolor[rgb]{0.73,0.47,0.47}{\bf{#1}}}
\newcommand{\hlkwb}[1]{\textcolor[rgb]{0.5,0.5,0.75}{\bf{#1}}}
\newcommand{\hlkwc}[1]{\textcolor[rgb]{0,0.5,0.75}{#1}}
\newcommand{\hlkwd}[1]{\textcolor[rgb]{0,0.27,0.4}{#1}}


\newcommand{\yslant}{0.5}
\newcommand{\xslant}{-0.6}
		 
% Titre		 
\title{Introduction � \LaTeX}
\author{Pascal Bessonneau}
\date{05/2016}




	
	














 				\subtitle{Faire des diapositives avec \LaTeX}

\begin{document}

  \begin{frame}
  \titlepage
  \end{frame}

  \begin{frame}<beamer>
    \frametitle{Plan}
    \tableofcontents
  \end{frame}	

\section{Les premiers pas sous Beamer}

		\begin{frame}[containsverbatim]
  		\frametitle{Les paquets pour faire des diapositives}

	Il existe deux paquets pour faire des paquets sous \LaTeX~:
\begin{description}
	\item[prosper] Nous n'en parlerons pas ici
	\item[beamer] c'est le paquet dont je vous parlerais ici
\end{description}
		
	Beamer a �t� utilis� pour faire ces diapositives. Vous pouvez donc regarder les sources des documents pour avoir un exemple plus vivant.

  	\end{frame}

		\begin{frame}[containsverbatim]
  		\frametitle{Dans le pr�ambule...}

	Premi�rement, il faut changer le type du document, au lieu de \emph{article} par exemple il faut mettre~:
\code
\documentclass{beamer}
	\usetheme{Warsaw}
\end{Verbatim}

	La premi�re ligne correspond au type \emph{beamer} qui nous int�resse ici. Le second permet de d�finir le th�me, c'est-�-dire l'apparence de vos diapos. Ce sont des "mod�les" de diapos. Vous pourrez �videmment personnaliser si vous le souhaitez.
		
  	\end{frame}


		\begin{frame}[containsverbatim]
  		\frametitle{Dans le corps du texte...}

Dans le corps du texte, les commandes � utiliser pour d�finir une diapositive sont les suivantes~:
\code
\begin{frame}
Mon texte...
\end{frame}
\end{Verbatim}

	ce qui donne la diapo de la page suivante...

  	\end{frame}

\begin{frame}
Mon texte...
\end{frame}


\section{Titres et organisation}

		\begin{frame}[containsverbatim]
  		\frametitle{Dans le corps du texte...}

	Les commandes \emph{\textbackslash section} et \emph{\textbackslash subsection} sont disponibles et vous permettent de structurer votre pr�sentation. 

\vspace{0.1cm}
Certains mod�les utilisent ces informations pour cr�er un affichage de la progression de la pr�sentation.

  	\end{frame}


		\begin{frame}[containsverbatim]
  		\frametitle{Titres des diapositives}

Vous pouvez rajouter des titres aux diapositives. Ces titres seront clairement visibles sur le document mais ne seront pas index�s dans la table des mati�res.

\vspace{0.1cm}
Sur ce mod�le il occupe le bandeau juste au dessus du texte. La syntaxe est relativement simple~:

  	\end{frame}


		\begin{frame}[containsverbatim]
  		\frametitle{Titres des diapositives}

\code
\begin{frame}
\frametitle{Mon titre}
Mon texte ....
\end{frame}
\end{Verbatim}

  	\end{frame}

\begin{frame}{Mon titre}
Mon texte ....
\end{frame}


		\begin{frame}[containsverbatim]
  		\frametitle{Titres des diapositives}

La syntaxe ci-dessous est aussi possible... pour le m�me r�sultat.

\code
\begin{frame}{Mon titre}
\frametitle
Mon texte ....
\end{frame}
\end{Verbatim}

  	\end{frame}


		\begin{frame}[containsverbatim]
  		\frametitle{Ajout de Verbatim}

S'il y a au moins un bloc \emph{listings} ou \emph{verbatim} il faut ajouter une balise pour \LaTeX~:

\code
\begin{frame}[containsverbatim]
\frametitle{Mon titre}
Mon texte ....
\end{frame}
\end{Verbatim}

  	\end{frame}


\section{Mise en page et environnements}

		\begin{frame}[containsverbatim]
  		\frametitle{Ajout de Verbatim}

On ne peut pas utiliser la syntaxe suivante dans cet ordre.

\code
\begin{frame}[containsverbatim]{Mon titre}
Mon texte ....
\end{frame}
\end{Verbatim}

  	\end{frame}


		\begin{frame}[containsverbatim]
  		\frametitle{Options de mise en page suppl�mentaires...}

Beamer offre la possibilit� de mettre tout ou partie du texte en �vidence en utilisant la balise \emph{block}

\code
\begin{frame}[containsverbatim]
\frametitle{Mon titre}
\begin{block}{Mon titre de bloc}
Mon texte ....
\end{block}
\end{frame}
\end{Verbatim}

  	\end{frame}

		\begin{frame}[containsverbatim]
  		\frametitle{Options de mise en page suppl�mentaires...}

Les environnements  suivants sont �galement disponibles...

\begin{itemize}
	\item alertblock
	\item theorem
	\item proof
	\item example
	\item beamercolorbox
\end{itemize}

 \end{frame}


		\begin{frame}[containsverbatim]
  		\frametitle{Options de mise en page suppl�mentaires...}

Comme d'autres logiciels, on peut ins�rer des �tapes~:

\code
\begin{itemize}
	\item<1-> alertblock
	\item<2-> theorem
	\item<3-> proof
	\item<4-> example
	\item<5-> beamercolorbox
\end{itemize}
\end{Verbatim}

 \end{frame}


		\begin{frame}[containsverbatim]
  		\frametitle{Options de mise en page suppl�mentaires...}

Dans le code pr�c�dent, \emph{alertblock} sera visible de la diapo un � la derni�re, \emph{theorem}, de la diapo 2 � la derni�re et ainsi de suite...

 \end{frame}

\section{Th�mes et sous forme d'article...}

		\begin{frame}[containsverbatim]
  		\frametitle{Les th�mes}

Pour avoir un aper�u des th�mes Beamer, il y a ce \href{http://mcclinews.free.fr/latex/beamergalerie.php}{site }.

 \end{frame}



		\begin{frame}[containsverbatim]
  		\frametitle{Sortie sous forme d'article}

Pour sortir les diapositives sous la forme d'un article, il suffit de changer le \emph{documentclass}~:

\code
\documentclass{article}
...
\usepackage{beamerarticle}
...
\end{Verbatim}


 \end{frame}


  					
\end{document}


