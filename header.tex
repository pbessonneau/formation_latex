\documentclass{beamer}
\usetheme[compress]{Singapore}
\useoutertheme{miniframes}

% Pour les documents en fran�ais...
	\usepackage[latin1]{inputenc}
	\usepackage[french]{babel}    
	\usepackage[french]{varioref} 
	
% Math�matiques
	\usepackage{amsmath}
	
% A documenter	
	\usepackage{moreverb}
	\usepackage{lipsum}

% Pour ins�rer des graphiques	
	\usepackage{eso-pic,graphicx}	% Graphique simples    
	\usepackage{subfigure}			% Graphiques multiples
	\usepackage{xcolor}
	\usepackage{tikz}
	\usetikzlibrary{positioning}						
	
% Pour ins�rer des couleurs	
	\usepackage{color}

% Outil suppl�mentaire pour les tableaux
	\usepackage{multirow}
 	\usepackage{booktabs}
	\usepackage{longtable}
	\usepackage{colortbl}
	
% Rotation des objets et des pages
	\usepackage{rotating}
	\usepackage{lscape}

% Pour ins�rer du code source, LaTeX ou SAS par exemple.
	\usepackage{verbatim}
	\usepackage{fancyvrb}
	\usepackage{listings} 

% Pour ins�rer des hyperliens
	\usepackage{hyperref}

% American Psychological Association (for bibliographic references).
	\usepackage{apacite}
  
% Pour l'utilisation des macros
	\usepackage{xspace}

% Pour l'utilisation de notes en fin de document.
	\usepackage{endnotes}

% Rotation
	\usepackage{rotating}

% Pour les t�ches de caf�
	\usepackage{coffee}

% Symboles suppl�mentaires
	\usepackage{bbding}
	\usepackage{pifont}

% Pour les listes num�rot�es
	\usepackage{enumerate}

% Pour la derni�re page
	\usepackage{lastpage}

% Pour Highlight d'Andre Simon
\usepackage{alltt}

% pour les symboles
\usepackage{keystroke}
%\usepackage{feyn}
\usepackage{bbding}
\usepackage{phonetic}

% Pour ins�rer des dessins de Linux
\newcommand{\LinuxA}{\includegraphics[height=0.5cm]{Graphiques/linux.png}}
\newcommand{\LinuxB}{\includegraphics[height=0.5cm]{Graphiques/linux.png}\xspace}

% Macro pour les petits dessins pour les diff�rents OS.
\newcommand{\Windows}{\emph{Windows}\xspace}
\newcommand{\Mac}{\emph{Mac OS X}\xspace}
\newcommand{\Linux}{\emph{Linux}\xspace}
\newcommand{\MikTeX}{MiK\tex\xspace}

% Des raccourcis pour les commandes \LaTeX, \TeX, ...
\newcommand{\latex}{\LaTeX\xspace}
\newcommand{\latexe}{\LaTeXe\xspace}
\newcommand{\tex}{\TeX\xspace}

% Commande pour le mode Verbatim
\newcommand{\code}{\vspace{0.2cm}\begin{Verbatim}[frame=single,label=Code,fontsize=\small]}
\newcommand{\tinycode}{\vspace{0.2cm}\begin{Verbatim}[frame=single,label=Code,fontsize=\tiny]}

% From Framabook (www.framasoft.net)
\newcommand{\latexcom}[1]{{\mdseries\ttfamily\upshape\symbol{92}#1}}
\newcommand{\indexcom}[1]{%
  \index{#1@\protect\texttt{\symbol{92}#1}}}
\newcommand{\ltxcom}[1]{%
  \latexcom{#1}\indexcom{#1}}  

\newcommand{\hlstd}[1]{\textcolor[rgb]{0,0,0}{#1}}
\newcommand{\hlnum}[1]{\textcolor[rgb]{0.5,0,0.5}{\bf{#1}}}
\newcommand{\hlesc}[1]{\textcolor[rgb]{1,0,1}{\bf{#1}}}
\newcommand{\hlstr}[1]{\textcolor[rgb]{0.65,0.52,0}{#1}}
\newcommand{\hlpps}[1]{\textcolor[rgb]{0,0,1}{#1}}
\newcommand{\hlslc}[1]{\textcolor[rgb]{0.95,0.47,0}{#1}}
\newcommand{\hlcom}[1]{\textcolor[rgb]{1,0.5,0}{#1}}
\newcommand{\hlppc}[1]{\textcolor[rgb]{0,0.5,0.75}{\bf{#1}}}
\newcommand{\hlopt}[1]{\textcolor[rgb]{1,0,0.5}{\bf{#1}}}
\newcommand{\hlipl}[1]{\textcolor[rgb]{0.62,0.36,1}{#1}}
\newcommand{\hllin}[1]{\textcolor[rgb]{0.19,0.19,0.19}{#1}}
\newcommand{\hlkwa}[1]{\textcolor[rgb]{0.73,0.47,0.47}{\bf{#1}}}
\newcommand{\hlkwb}[1]{\textcolor[rgb]{0.5,0.5,0.75}{\bf{#1}}}
\newcommand{\hlkwc}[1]{\textcolor[rgb]{0,0.5,0.75}{#1}}
\newcommand{\hlkwd}[1]{\textcolor[rgb]{0,0.27,0.4}{#1}}


\newcommand{\yslant}{0.5}
\newcommand{\xslant}{-0.6}
		 
% Titre		 
\title{Introduction � \LaTeX}
\author{Pascal Bessonneau}
\date{05/2016}




	
	












